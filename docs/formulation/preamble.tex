%-------------------- 
%- commands and packages used by the authors
%--------------------


%package for spacing around custom commands
\usepackage{xspace}
\usepackage{tabularx}  %load tabularx before arydshln
%\usepackage{arydshln}



%Everything below was copied from the Math.tex file from Mark until noted
\usepackage{amsmath,epsf,amssymb}
\usepackage{setspace}
\usepackage{color}
\usepackage{lineno}
\usepackage[usenames,dvipsnames]{xcolor}
\usepackage{fancyvrb}
\usepackage{url}
\usepackage{wasysym}
\usepackage{longtable}
%\usepackage{floatrow}
\usepackage{lipsum}
\usepackage{adjustbox}
\usepackage{cite}
\usepackage[acronym]{glossaries}
\usepackage [english]{babel}
%\usepackage [autostyle, english = american]{csquotes}
\usepackage{enumitem}
\usepackage{indentfirst}
\usepackage{relsize}
%\MakeOuterQuote{"}
\usepackage{relsize}
%End of copied from the Math.tex file
\usepackage{cancel}
\usepackage[normalem]{ulem}
%-use solarreserve as a command just in case we decide to scrub later.
\newcommand{\sampt}{MSPT\xspace}
\newcommand{\abbr}[2]{#2 (#1)}
\newcommand{\tcr}{\textcolor{red}}
\newcommand{\tcb}{\textcolor{blue}}
\newcommand{\tcy}{\textcolor{yellow}}
\newcommand{\tcg}{\textcolor{green}}
\newcommand{\tcp}{\textcolor{purple}}
\newcommand{\tcc}{\textcolor{cyan}}
\newcommand{\scotu}{s-CO$_2$\xspace}
\newcommand{\TM}{\textsuperscript{\tiny TM}}
\newcommand{\mc}[1]{\mathcal{#1}}
%\newcommand{\munderbar}[1]{\underaccent{\bar}{#1}}

\newcommand{\sr}{{\xspace}SolarReserve\xspace}
\newcommand{\sd}{{\xspace}PS\xspace}
\newcommand{\crescentdunes}{{\xspace}Crescent Dunes\xspace}
\newcommand{\rice}{{\xspace}Rice\xspace}

\newif\ifsrok
\srokfalse
\sroktrue

%\ifsrok\else
%\renewcommand{\sr}{{\xspace}Company-X\xspace}
%\renewcommand{\sd}{{\xspace}Iterative Optimizer\xspace}
%\renewcommand{\crescentdunes}{{\xspace}Plant A\xspace}
%\renewcommand{\sandstone}{{\xspace}Plant B\xspace}
%\fi

\newcommand{\sse}{$_\textsf{e}$}
\newcommand{\sst}{$_\textsf{t}$}

%\renewcommand{\tcr}[1]{}

%\DeclareCaptionLabelFormat{andtable}{#1~#2  \&  \tablename~\thetable}

\newcommand{\figref}[1]{Figure~\ref{#1}\xspace}
\newcommand{\tabref}[1]{Table~\ref{#1}\xspace}

\graphicspath{{./figures/}}    % all figures to go in there!
%\usepackage[colorlinks, linkcolor=black, citecolor=black, urlcolor=black]{hyperref}
\usepackage{multirow}	%table cells spanning multiple rows
%\usepackage{amsmath}   % Turned off due to warning
\usepackage{amssymb,wasysym}  %Math fonts
\usepackage{verbatim}

\usepackage{longtable}
\usepackage{float}
\restylefloat{table}	% use the H placement option to make sure tables do not get repositioned
%\usepackage{subcaption}
\usepackage[caption=false]{subfig}	%allow subfigures

%--- Table column commands with specified widths:
%\usepackage{array}
% L		left-aligned
% C		centered
% R		right-aligned
\newcolumntype{L}[1]%
{>{\raggedright\let\newline\\\arraybackslash\hspace{0pt}}m{#1}}
\newcolumntype{C}[1]%
{>{\centering\let\newline\\\arraybackslash\hspace{0pt}}m{#1}}
\newcolumntype{R}[1]%
{>{\raggedleft\let\newline\\\arraybackslash\hspace{0pt}}m{#1}}
\newcolumntype{T}[1]%
{>{\raggedright\let\newline\\\arraybackslash\hspace{0pt}}p{#1}}

\newcommand{\rcgray}[1]{\rowcolor{Gray!40}\multicolumn{3}{c}{\textit{#1}}}
%\newcommand\norm[1]{\left\lVert#1\right\rVert}

%Adding accents package for underbar
%\usepackage{etoolbox}
%\usepackage{accents}
%\robustify{\underaccent}
%\MakeRobust{\underaccent} % make \underaccent not fragile in moving arguments
%\newcommand{\ubarw}{\underaccent{\bar}{w}}
%\newcommand{\ubarf}{\underaccent{\bar}{f}}
%\newcommand{\ubarb}{\underaccent{\bar}{b}}
\newcommand{\ubarw}{\underline{w}}
\newcommand{\ubarf}{\underline{f}}
\newcommand{\ubarb}{\underline{b}}
\newcommand{\reoptlitemod}{$\mathcal{R}$}
\newcommand{\resetmod}{$\mathcal{B}$}
\newcommand{\resetupper}{$\bar{\mathcal{B}}$}
%\newcommand{\resetlower}{{$\underaccent{\bar}{\mathcal{B}}$}}
\newcommand{\resetlower}{{$\underline{\mathcal{B}}$}}
\newcommand{\Ell}{\mathcal{L}}



\usepackage{tikz}
\usepackage{varwidth}
\usetikzlibrary{positioning, automata}
\newcommand{\mwnote}[2][]{
\begin{tikzpicture}[remember picture]
\node[font=\small\sffamily, fill=yellow, opacity=0.9, draw=brown, rectangle, yshift=0in, xshift=0in] {\begin{varwidth}{2.5in}\textcolor{red}{#1}\par#2\end{varwidth}};
\end{tikzpicture}
}
%for final, uncomment the following
%\renewcommand{\mwnote}[2][]{\xspace}